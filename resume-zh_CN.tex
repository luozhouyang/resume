% !TEX TS-program = xelatex
% !TEX encoding = UTF-8 Unicode
% !Mode:: "TeX:UTF-8"

\documentclass{resume}
\usepackage{zh_CN-Adobefonts_external} % Simplified Chinese Support using external fonts (./fonts/zh_CN-Adobe/)
% \usepackage{NotoSansSC_external}
% \usepackage{NotoSerifCJKsc_external}
% \usepackage{zh_CN-Adobefonts_internal} % Simplified Chinese Support using system fonts
\usepackage{linespacing_fix} % disable extra space before next section
\usepackage{cite}

\begin{document}
\pagenumbering{gobble} % suppress displaying page number

\name{罗周杨}

\basicInfo{
  \email{zhouyang.luo@gmail.com} \textperiodcentered\ 
  \phone{(+86) 133-8153-6323} \textperiodcentered\ 
  \github{https://github.com/luozhouyang}}
 
\section{\faGraduationCap\  教育背景}
\datedsubsection{\textbf{上海大学}}{2013.09 -- 2017.07}
\textit{本科,学士}\ 通信学院电子信息工程

\section{\faCogs\ 个人技能}
% increase linespacing [parsep=0.5ex]
\begin{itemize}[parsep=0.5ex]
  \item 熟悉NLP相关技术,包括不限于TF-IDF、TextRank、Word2Vec、ELMo、Doc2Vec、Transformer、BERT等算法或模型,熟悉Seq2Seq架构和Attention机制,熟悉TensorFlow和Keras框架
 \item 熟悉Java和Python,熟悉常用设计模式,熟悉Spring Boot、Flask、FastAPI等Web框架,有一定的Hadoop、Hive、PySpark使用经验
 \item 熟悉Linux环境,熟悉Docker、Git、Vim、Gradle等工具的使用
\end{itemize}

\section{\faCode\ 开源项目}

\datedsubsection{\textbf{transformers-keras} }{\href{https://github.com/luozhouyang/transformers-keras}{https://github.com/luozhouyang/transformers-keras}} 
\begin{itemize}
 \item 基于Transformer的多个模型实现,包括Transformer、BERT和ALBERT。
 \item 使用tf.data API构建数据输入管道,实现模型的导出部署,实现了加载预训练模型功能。
\end{itemize}

\datedsubsection{\textbf{python-string-similarity}  400+ star }{\href{https://github.com/luozhouyang/python-string-similarity}{https://github.com/luozhouyang/python-string-similarity}} 
\begin{itemize}
 \item 实现多种字符串相似度的计算方法,包括最长公共子序列、Jaccard距离、编辑距离等。
\end{itemize}

\section{\faUsers\ 工作经历}
\datedsubsection{\textbf{e成科技} |  \href{https://ifchange.com}{https://ifchange.com} \hspace{10mm} NLP算法工程师}{2018.12 -- 至今} 

项目:搜索之Query解析
\begin{itemize}
  \item 将用户输入的Query进行解析,包括Query纠错、意图识别、Query Weighting、Query扩展等模块。
 \item 使用ngram语言模型,通过语言复杂度(PPL)进行错误检测,使用同音字、同型字和混淆集等词典的方式进行纠错。
 \item 通过NER接口得到Query中的职能、行业、学历等实体,通过策略计算得到Query不同Term的权重,得到多个意图,然后进行整合得到最终的主意图。
 \item 通过规则和Embedding相似度的方式,对主意图的行业和职能等实体进行扩展,得到扩展意图。基于Embedding相似度的扩展使用该相似度作为扩展意图的权重。
\end{itemize}

项目:文本相关性
\begin{itemize}
  \item 基于BERT模型计算JD和CV之间的文本相关性,以提升推荐的效果。
  \item 使用自己的千万级别的JD和CV的文本内容整理得到预训练数据。通过规则删除一些不重要的句子以保证序列长度不超过512,中文使用按字分词以减小词典。
  \item 保持BERT预训练的MLM任务不变,NSP任务替换成JD和CV是否匹配的二分类任务。二分类任务上验证集准确率为0.90、F1为0.92。二分类任务的得分直接作为文本相关性得分。
\end{itemize}

项目:搜索、推荐工程的迭代和优化
\begin{itemize}
  \item 负责相似CV的推荐业务,完成并优化Elastic Search的Query DSL构建,完成并优化CV之间的特征提取、匹配分数计算和候选集排序。
  \item 优化推荐工程性能,精简Elastic Search的Query构建以减小候选集和提升查询速度,同时分片并行请求耗时的外部服务调用、并行计算候选集的特征和匹配得分。服务的平均响应时间提升3倍。
\end{itemize}

项目:关键词抽取
\begin{itemize}
 \item TF-IDF作为关键词抽取的Baseline,但是存在词语权重区分不够明显的问题。
  \item 从预训练BERT模型中,抽取出字的Attention权重矩阵,按列合并然后归一化,获得词语的权重。
  \item 然后使用基于Sentence Embedding的EmbedRank模型进行得到关键词及其权重。
  \item 线性加权三种权重得到最终的关键词及其权重。
\end{itemize}

\datedsubsection{\textbf{51job}  |  \href{https://51job.com}{https://51job.com} \hspace{20mm} NLP算法工程师}{2017.07 -- 2018.12}
\begin{onehalfspacing}
 项目:文本分词
 \begin{itemize}
  \item 尝试使用jieba分词,发现对于道路文本分词效果较差。使用规则从预料中整理得到一批词典,使用jieba + 自定义词典 的方式有效提升了分词的准确性。但词典之外的样本分词效果依然不佳。
  \item 为了充分利用上下文来达到更好的分词效果,实现基于BiLSTM + CRF的神经网络分词模型,通过jieba + 词典的方式构建大部分训练样本,人工标注少量无法通过前者正确分词的样本,最终得到大约300万的训练数据,最终的分词效果提升非常明显,F1达到0.95。
  \item 代码开源 \href{https://github.com/luozhouyang/deepseg}{https://github.com/luozhouyang/deepseg}
\end{itemize}

项目:地址文本纠错
\begin{itemize}
  \item 使用字符级别的ngram语言模型来进行错误检测,对于每个地址文本,计算2-gram、3-gram的语言困惑度(PPL),对于高于平均值的ngram标记为疑似错误的token。PPL得分计算通过使用自己数据集训练的KenLM模型得到。
  \item 使用同音字、同形字和混淆集词典,替换疑似错误的token,计算文本困惑度(PPL),选择PPL最低的序列作为纠正文本。
  \item 为了让纠错过程能考虑更长的上下文信息,使用基于RNN的Seq2Seq + Attention的生成式模型来实现纠错,使用300万左右的数据训练模型,正负样本比例大约5 : 6,模型在测试集准确率(Accuracy)达到0.86。
  \item 模型使用Tensorflow Serving + Docker的形式部署,使用Beam Search方式解码,平均推断时间50ms(GPU)。
\end{itemize}
\end{onehalfspacing}



%\section{\faInfo\ 其他}
% increase linespacing [parsep=0.5ex]
%\begin{itemize}[parsep=0.5ex]
%  \item 博客\hspace{4mm}: \href{https://luozhouyang.github.io}{https://luozhouyang.github.io}
%\end{itemize}

%% Reference
%\newpage
%\bibliographystyle{IEEETran}
%\bibliography{mycite}
\end{document}
