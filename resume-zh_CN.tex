% !TEX TS-program = xelatex
% !TEX encoding = UTF-8 Unicode
% !Mode:: "TeX:UTF-8"

\documentclass{resume}
\usepackage{zh_CN-Adobefonts_external} % Simplified Chinese Support using external fonts (./fonts/zh_CN-Adobe/)
% \usepackage{NotoSansSC_external}
% \usepackage{NotoSerifCJKsc_external}
% \usepackage{zh_CN-Adobefonts_internal} % Simplified Chinese Support using system fonts
\usepackage{linespacing_fix} % disable extra space before next section
\usepackage{cite}

\begin{document}
\pagenumbering{gobble} % suppress displaying page number

\name{罗周杨}

\basicInfo{
  \email{zhouyang.luo@gmail.com} \textperiodcentered\ 
  \phone{(+86) 133-8153-6323} \textperiodcentered\ 
  \github{https://github.com/luozhouyang}}
 
\section{\faGraduationCap\  教育背景}
\datedsubsection{\textbf{上海大学}}{2013.09 -- 2017.07}
\textit{本科,学士}\ 通信学院电子信息工程

\section{\faUsers\ 工作经历}
\datedsubsection{\textbf{e成科技} |  \href{https://ifchange.com}{https://ifchange.com} \hspace{10mm} NLP算法工程师}{2018.12 -- 2020.06} 
项目:JD和CV的文本相关性
\begin{itemize}
  \item 使用千万级别的JD和CV的文本内容,经过分句和筛选处理,整理得到预训练数据。把BERT预训练的NSP任务替换成JD和CV是否匹配的二分类任务。然后进行预训练,匹配任务的训练集准确率达到0.94,测试集准确率0.92,比之前线上的效果提升3个点左右。
\end{itemize}

项目:JD和CV的关键词抽取
\begin{itemize}
  \item 从上述预训练BERT模型中,抽取出Attention权重作为词语的权重,特殊Token的权重不小,并且不同层的Attention权重差别较大。对比之后选择最后一层,并且去除特殊Token。
  \item 使用基于Sentence Embedding的EmbedRank模型进行关键词抽取,线性组合两者权重,对比之前线上效果,无意义词语显著减少,重要词语的权重区分更明显。
\end{itemize}

项目:细分领域的职能识别
\begin{itemize}
  \item 在上述预训练的BERT模型之上,构建NER模型进行微调。解决大部分细分领域(通用的NER覆盖不到)的职能识别不准的问题,显著提高了CV的召回,以及JD和CV之间的推荐度。
\end{itemize}

项目:JD和CV的推荐工程
\begin{itemize}
  \item 优化推荐工程性能,并行请求ES和处理CV,响应时间提升5倍。优化所涉及的基础算法服务。
\end{itemize}

\datedsubsection{\textbf{51job}  |  \href{https://51job.com}{https://51job.com} \hspace{20mm} 软件开发工程师}{2017.07 -- 2018.12}
\begin{onehalfspacing}
项目:名片地址文本纠错
\begin{itemize}
  \item 为了解决jieba对道路等文本分词效果不佳的问题,实现BiLSTM + CRF的神经网络分词模型,对于地址文本的分词效果提升明显。\href{https://github.com/luozhouyang/deepseg}{https://github.com/luozhouyang/deepseg}
  \item 使用基于RNN的Seq2Seq + Attention的生成式模型来实现纠错,测试集准确率达到0.86。使用Tensorflow Serving + Docker部署模型,平均推断时间50ms(GPU)。
\end{itemize}
\end{onehalfspacing}


\section{\faCode\ 开源项目}
\begin{itemize}[parsep=0.5ex]
  \item \href{https://github.com/luozhouyang/python-string-similarity}{luozhouyang/python-string-similarity}:多种字符串相似度的衡量算法
  \item \href{https://github.com/luozhouyang/transformers-keras}{luozhouyang/transformers-keras}:基于Transformer的多个模型实现
 \item \href{https://github.com/luozhouyang/embedrank}{luozhouyang/embedrank}:基于Sentence Embedding的无监督关键词抽取
\end{itemize}

\section{\faCogs\ 个人技能}
% increase linespacing [parsep=0.5ex]
\begin{itemize}[parsep=0.5ex]
  \item 编程语言:3年Python使用经验,作为脚本和构建深度学习模型,有使用Flask、FastAPI构建Web服务的经验;4年Java使用经验,熟悉多线程和常见的设计模式,有Spring Boot构建Web服务经验;2年Android应用开发经验(大学期间,Java语言)
 \item 工具相关:常年在Linux环境下工作,熟悉Docker使用,熟悉Git及其协作
\end{itemize}

\section{\faInfo\ 其他}
% increase linespacing [parsep=0.5ex]
\begin{itemize}[parsep=0.5ex]
  \item GitHub: \href{https://github.com/luozhouyang}{https://github.com/luozhouyang}
  \item 博客\hspace{4mm}: \href{https://luozhouyang.github.io}{https://luozhouyang.github.io}
\end{itemize}

%% Reference
%\newpage
%\bibliographystyle{IEEETran}
%\bibliography{mycite}
\end{document}
